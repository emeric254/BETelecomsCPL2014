
\chapter*{Introduction}
    \paragraph{}
        ...
\addcontentsline{toc}{part}{Introduction}


\chapter*{Problématique}
    \paragraph{}
        En quoi les CPL peuvent-ils faciliter le développement des objets connectés dans les maisons individuelles ?

\addcontentsline{toc}{part}{Problématique}


\chapter{La Domotique et les objets connectés}
    \section{Qu’est-ce que c’est ?}
        \subsection{Définition du CPL}
            \paragraph{Courant porteur ligne}
La transmission des données est atténuée par différents critères comme la vétusté du réseau électrique, la longueur des câbles, l'utilisation de multiprises, la présence d'autres appareils perturbant le signal ou encore la qualité des adaptateurs CPL utilisés.
Néanmoins, le CPL est une alternative très intéressante aux câbles disgracieux et au Wi-Fi pas toujours très sécurisé.
Créer son réseau local et partager sa connexion haut-débit devient alors aussi simple que de brancher n'importe quel appareil sur une prise électrique murale.
            \paragraph{Principes de cette technologie}
La technologie du courant porteur en ligne (CPL) permet de transporter un signal de haute fréquence en le superposant au signal 50hz du courant électrique qui est délivré par les traditionnelles prises de courant.
Grâce à la transmission d'informations numériques sur le réseau électrique 220 Volts existant, il est ainsi possible de créer un réseau local Internet haut-débit avec le réseau électrique d'un logement.
S'appuyant sur les câbles électriques déjà présents dans la plupart des habitations, le CPL permet la mise en réseau d'ordinateurs et de matériels équipés d'une prise RJ45 (Ethernet), dans un appartement, un bureau ou une maison, avec un débit théorique de transfert des données de l'ordre de plusieurs centaines de mégabits par seconde.
            \paragraph{Normes et hétérogénéité}
                \begin{itemize}
                    \item 14 Mbits/s \emph{(dépassé)} ;
                    \item 85 Mbits/s \emph{(dépassé)} ;
                    \item 200 Mbits/s \emph{(le strict minimum)} ;
                    \item 500 Mbits/s ;
                    \item 600 Mbits/s ;
                    \item 650 Mbits/s \emph{range+ chez Devolo} ;
                    \item 1 Gbit/s (1000 Mbits/s) \emph{avec prises de terre uniquement}.
                \end{itemize}
Tout comme pour les débits WiFi, \emph{ le \emph{802.11n} étant limité pour l'instant à 300 Mbits/s, le \emph{802.11ac} en double-bande aux alentours de 1900 Mbits/s}, les débits annoncés (\emph{théoriques}) pour le courant porteur en ligne ne sont pas aussi importants dans la réalité.
Ainsi, les vitesses réellement observées varient globalement de 45 à 75\% par rapport aux débits annoncés.
Pour un adaptateur CPL 600 Mbits/s, cela permet tout de même des débits moyens tournant autour des 450 Mbits/s, soit près de 60 Mio/s.

        \subsection{Définition de la domotique}
            \paragraph{Services automatisés}
                ...
            \paragraph{Maison plus confortable}
                ...
    \section{Etude de l’art}
        ...
    \section{Développement du CPL}
        ...

\chapter{Produits actuellement disponibles}

\chapter{Exemple type}
