
\chapter*{Introduction}
    \paragraph{}
Dans le cadre de formation Système Télécommunication et Réseaux Informatiques, nous avons du réaliser un bureau d'étude en télécommunication.
Le but de ce BE est de savoir faire une analyse d'un problème, savoir écrire une problématique, un planning prévisionnel des taches, faire une synthèse du travail réalisé et effectuer des recherches sur une problématique.
Nous devons aussi nous préparer à répondre a des questions d'un public sur notre thème de recherche.
    \paragraph{}
Notre thème est : « Les courants porteurs de ligne »
Dans ce diaporama nous allons vous expliquer ce qu'est un CPL, ses usages et ses contraintes, ainsi qui est proposé actuellement  pour finir nous allons vous faire une étude de marché.
    \paragraph{}
De nos jours nous voyons des maisons équipées avec de plus en plus d'objets et d'automatismes connectés, c'est la domotique.
Elle représente le croisement entre la recherche en électronique, automatisme et celle du bâtiment voire génie civil.
Et a pour but de centraliser et d'automatiser les taches qui sont présentes dans un bâtiment, ici une maison individuelle.
    \paragraph{}
Comme toute avancé technologique, celle de la domotique vise à améliorer et faciliter la vie de son utilisateur.
    \paragraph{}
Cette technologie utilisant un certain nombre d'appareils pouvant être vu comme des terminaux distribuant de la donnée, il est nécessaire de les interconnecter afin qu'ils puissent échanger avec l'unité centrale et entre eux.
    \paragraph{}
Suivant le type de construction, sa vétusté, sa disposition, la mise en place d'un tel système peut s'avérer très complexe, mais des solutions existent, c'est là qu'intervient le CPL.
    \paragraph{}
Nous allons montrer comment, le CPL peuvent représenter un avenir dans le domaine de la domotique en maison individuelle.

\addcontentsline{toc}{part}{Introduction}



\chapter*{Problématique}
    \paragraph{}
    \paragraph{}
        \center{\LARGE En quoi les CPL peuvent-ils faciliter le développement des objets connectés dans les maisons individuelles ?}

\addcontentsline{toc}{part}{Problématique}



\chapter{La Domotique et les objets connectés}
    \section{La Domotique}
        \subsection{Qu’est-ce que c’est ?}
            \paragraph{}
La domotique est l’ensemble des techniques de l'électronique, de physique du bâtiment, d'automatisme, de l'informatique et des télécommunications utilisées dans les bâtiments, plus ou moins « interopérables » et permettant de centraliser le contrôle des différents systèmes et sous-systèmes de la maison et de l'entreprise (chauffage, volets roulants, porte de garage, portail d'entrée, prises électriques, etc.).
La domotique vise à apporter des solutions techniques pour répondre aux besoins de confort (gestion d'énergie, optimisation de l'éclairage et du chauffage), de sécurité (alarme) et de communication (commandes à distance, signaux visuels ou sonores, etc.) que l'on peut retrouver dans les maisons, les hôtels, les lieux publics, etc.
            \paragraph{}
le pilotage des appareils « électrodomestiques », électroménagers par programmation d'horaires et/ou de macro (suites d'actions programmées réalisées par les appareils électroménagers) définis par l'usager. Le déclenchement des appareils peut être aussi lié à des évènements (détecteurs de mouvement, télécommandes, etc.) ;
la gestion de l'énergie, du chauffage (par exemple, il est possible de gérer les apports naturels (calories, frigories, vent, lumière, eau…) en fonction de l'enveloppe thermique du bâtiment), de la climatisation, de la ventilation, de l'éclairage, de l’ouverture et de la fermeture des volets (en fonction de l'ensoleillement ou de l'heure de la journée, par exemple), de l'eau (le remplissage de la baignoire peut s’arrêter automatiquement grâce à un senseur, les robinets de lavabos peuvent ouvrir l’eau à l’approche des mains, etc.). Il est également possible de recharger certains appareils électriques (ordinateurs, véhicules électriques, etc.) en fonction du tarif horaire (voir Smart grid). Un compteur communicant peut être intégré dans un smart-grid et/ou raccordé à un système de télégestion. La Régulation/programmation du chauffage permet d'importantes économies ;
la sécurité des biens et des personnes (alarmes, détecteur de mouvement, interphone, digicode) ;
la communication entre appareil et utilisateur par le biais de la « sonification » (émission de signaux sous forme sonore) ;
le « confort acoustique ». Il peut provenir de l'installation d'un ensemble de haut-parleurs permettant de répartir le son et de réguler l’intensité sonore ;
la compensation des situations de handicap et de dépendance.
Elle peut aussi chercher à diminuer son empreinte écologique (« éco-domotique ») et celle de ses utilisateurs par une éco conception, en facilitant une meilleure maîtrise de la consommation énergétique de l'habitat, en améliorant l'efficience énergétique des installations
            \paragraph{}
Avec le temps, la domotique tend à sortir de la maison. Elle met par exemple en relation des unités d'habitation entre elles et avec un immeuble (c'est l'immotique) et avec la ville (on entre alors dans l'« urbatique » et/ou avec un gestionnaire / propriétaire et/ou d'autres entités fournissant par exemple des services (eau, énergie, livraison de nourriture, soins à domicile ou distant, lavage de vêtements, etc). Si ces services visent prioritairement à moins dégrader l'environnement, on parle parfois d'« éco domotique urbaine ».
            \paragraph{}
Ces coûts proviennent essentiellement pour 60 \% de la partie électrique (main d’œuvre électricien, nombres d’éléments à contrôler, nombre de capteurs, nombre de pièces, etc.) et pour environ 40 % de la partie chauffage (nombre de pièces à chauffer, nombre de radiateurs, etc.). Par ailleurs, le coût d'évolution est plus élevé, des travaux importants pouvant être nécessaire pour compléter le réseau.
            \paragraph{Exemples de scénarios :}
            %    \begin{figure}
            %        \begin{center}
                        %%% au choix
                        %\includegraphics{image.png}
                        %\includegraphics[height=128, width=128]{image.png}
                        %\includegraphics[scale=0.5]{image.png}
            %        \end{center}
            %            \caption{ Laule } % ce qui apparait juste en dessous de l'image
            %            \label{c'est styler !}
            %    \end{figure}
en partant au travail, un simple clic sur un interrupteur installé dans l’entrée enclenche le scénario « départ au travail ».
L’éclairage s’éteint, le garage s’ouvre, le chauffage se met en veille au bout de 15 minutes, les volets et le garage se ferment après 30 minutes ;
en quittant le travail pour rentrer chez soi, on actionne le scénario de retour à l’aide du téléphone WAP ou depuis l’ordinateur du bureau : les volets s’ouvrent et le chauffage passe en mode confort ;
quand on est fatigué, on agit sur la télécommande de la maison afin d’enclencher le scénario « relaxation », les lumières se tamisent, un fond sonore apaisant se propage dans la pièce.


    \section{CPL}
        \subsection{Définition du CPL}
            \paragraph{Courant porteur ligne}
La transmission des données est atténuée par différents critères comme la vétusté du réseau électrique, la longueur des câbles, l'utilisation de multiprises, la présence d'autres appareils perturbant le signal ou encore la qualité des adaptateurs CPL utilisés.
Néanmoins, le CPL est une alternative très intéressante aux câbles disgracieux et au Wi-Fi pas toujours très sécurisé.
Créer son réseau local et partager sa connexion haut-débit devient alors aussi simple que de brancher n'importe quel appareil sur une prise électrique murale.
            \paragraph{Principes de cette technologie}
La technologie du courant porteur en ligne (CPL) permet de transporter un signal de haute fréquence en le superposant au signal 50hz du courant électrique qui est délivré par les traditionnelles prises de courant.
Grâce à la transmission d'informations numériques sur le réseau électrique 220 Volts existant, il est ainsi possible de créer un réseau local Internet haut-débit avec le réseau électrique d'un logement.
S'appuyant sur les câbles électriques déjà présents dans la plupart des habitations, le CPL permet la mise en réseau d'ordinateurs et de matériels équipés d'une prise RJ45 (Ethernet), dans un appartement, un bureau ou une maison, avec un débit théorique de transfert des données de l'ordre de plusieurs centaines de mégabits par seconde.
            \paragraph{Normes et hétérogénéité}
                \begin{itemize}
                    \item 14 Mbits/s \emph{(dépassé)} ;
                    \item 85 Mbits/s \emph{(dépassé)} ;
                    \item 200 Mbits/s \emph{(le strict minimum)} ;
                    \item 500 Mbits/s ;
                    \item 600 Mbits/s ;
                    \item 650 Mbits/s \emph{range+ chez Devolo} ;
                    \item 1 Gbit/s (1000 Mbits/s) \emph{avec prises de terre uniquement}.
                \end{itemize}
Tout comme pour les débits WiFi, \emph{ le \emph{802.11n} étant limité pour l'instant à 300 Mbits/s, le \emph{802.11ac} en double-bande aux alentours de 1900 Mbits/s}, les débits annoncés (\emph{théoriques}) pour le courant porteur en ligne ne sont pas aussi importants dans la réalité.
Ainsi, les vitesses réellement observées varient globalement de 45 à 75\% par rapport aux débits annoncés.
Pour un adaptateur CPL 600 Mbits/s, cela permet tout de même des débits moyens tournant autour des 450 Mbits/s, soit près de 60 Mio/s.

        \subsection{Définition de la domotique}
            \paragraph{Services automatisés}
                ...
            \paragraph{Maison plus confortable}
                ...
    \section{Etude de l’art}
        ...
    \section{Développement du CPL}
        ...



\chapter{Produits actuellement disponibles}



\chapter{Exemple type}
