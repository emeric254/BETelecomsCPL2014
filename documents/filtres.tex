\documentclass[a4paper,11pt]{article}
\usepackage[utf8]{inputenc}
\usepackage{color,amsmath,xcolor,listings,graphicx}
\usepackage[francais]{babel}

%% paramétrage pour les zones de 'code'
\lstdefinelanguage{javascript}{
  keywords={typeof, new, true, false, catch, function, return, null, catch, switch, var, if, for, in, while, do, else, case, break},
  keywordstyle=\color{blue}\bfseries,
  ndkeywords={class, export, boolean, throw, implements, import, this},
  ndkeywordstyle=\color{darkgray}\bfseries,
  identifierstyle=\color{black},
  sensitive=false,
  comment=[l]{//},
  morecomment=[s]{/*}{*/},
  commentstyle=\color{purple}\ttfamily,
  stringstyle=\color{red}\ttfamily,
  morestring=[b]',
  morestring=[b]"
}

\lstset{
    language=javascript, commentstyle=\textit, frame=shadowbox,
    rulesepcolor=\color{gray}, basicstyle=\ttfamily\small, columns=flexible,
    tabsize=4, extendedchars=true, showspaces=false,
    showstringspaces=false, numbers=left, numberstyle=\tiny,
    breaklines=true, breakautoindent=true, captionpos=b, morecomment=[l]{//}
%language=Octave %-> choose the language of the code
%basicstyle=\footnotesize %-> the size of the fonts used for the code
%numbers=left %-> where to put the line-numbers
%numberstyle=\footnotesize %-> size of the fonts used for the line-numbers
%stepnumber=2 -> the step between two line-numbers.
%numbersep=5pt -> how far the line-numbers are from the code
%backgroundcolor=\color{white} -> sets background color (needs package)
%showspaces=false -> show spaces adding particular underscores
%showstringspaces=false -> underline spaces within strings
%showtabs=false -> show tabs within strings through particular underscores
%frame=single -> adds a frame around the code
%tabsize=2 -> sets default tab-size to 2 spaces
%captionpos=b -> sets the caption-position to bottom
%breaklines=true -> sets automatic line breaking
%breakatwhitespace=false -> automatic breaks happen at whitespace
%morecomment=[l]{//} -> displays comments in italics (language dependent)
}


%% infos du document
\title{Filtres}
\author{Loïc Barbaresco, Rémi Barbaste, Robin Degironde, Émeric Tosi}
\date{\today}


\begin{document}

%% Afficher la page de garde : Titre + Auteur(s) + Date de dernière compilation
    \maketitle{}
    \clearpage

    \setcounter{tocdepth}{1} % définir la profondeur de l'Index
    \renewcommand{\contentsname}{Sommaire} % renommer l'Index en Sommaire
    \tableofcontents{} % afficher l'Index
    \clearpage


\section{Définition}
    \paragraph{}
Un filtre est un dispositif (actif ou passif) qui permet de réaliser une opération de traitement du signal.
Il permet en effet de transformer le signal reçu en entrée en un signal de sortie différent par le biais de circuits électroniques qui servent à modifier le spectre de fréquence et/ou la phase qui lui est associé.
Ils ont pour but d'isoler,  d'éliminer ou de séparer des signaux situés dans une certaine bande passante.
Les opérations de traitements peuvent consister à éliminer ou affaiblir des fréquences parasites (comme du bruit) et/ou à isoler une information utile présente au sein d'une certaine bande passante d'un signal.

    \clearpage

\section{Différences entre filtre analogique et numérique}
    \paragraph{}
Parmi les différents filtres existants, on peut les classer selon deux catégories : ceux analogiques et ceux numériques.
    \paragraph{}
Tout d'abord, on parle de filtre analogique lorsqu'un signal qui traverse un circuit électrique est différent en sortie que celui présent en entrée et que cette modification est désiré par l'utilisateur.
    \paragraph{}
Ensuite,
    \paragraph{}
==> Définition d'un filtre analogique
    \paragraph{}
==> Définition d'un filtre numérique


\section{Différences entre filtre passif et filtre actif}
    \paragraph{}
\emph{Aborder notamment le problème posé par l'utilisation des HF.}
Voir page 13 du doc 'filtrage-analogique.pdf'.
    \paragraph{}
Le problème posé par l'utilisation des hautes fréquences (HF) est que la fonction d'amplificateur opérationnel n'existe pas (ecrit dans cours) .....
    \paragraph{}
Comparaison entre les lois mathématiques des filtres de Butterworth, Tchebyscheff, Cauer,Bessel et Legendre.
    \paragraph{}
Avantages, inconvénients, .....
    \paragraph{}
Choisir des critères de comparaison, les catalogues sont interdits.

    \clearpage

\section{Caracteristiques}
    \paragraph{}
Ordre plus petit pour une grande sélectivité
    \paragraph{}
Pente d’atténuation supérieure à 40dB/décade autour de la fréquence de coupure
    \paragraph{}
À ordre n, il présente n ondulation(s) dans la bande passante
    \paragraph{}
Filtres d’ordre impair : Impédances d’entrée et de sortie identiques


\section{Inconvenients}
    \paragraph{}
Temps de propagation de groupe non constant en bande passante (déphasage non linéaire)
    \paragraph{}
Très mauvaise linéarité de phase
    \paragraph{}
Ondulation dans la bande passante

    \clearpage

\section{Calculs}
\subsection{calcul de la pulsation}
    \paragraph{}
    ...\[ \mbox{Pulsation Wc} = 2 * \pi * \mbox{ Fréquence de coupure}\]
    \begin{lstlisting}
        /* pulsation */

        Wc = 2 * Math.PI * freqCoup;

    \end{lstlisting}

\subsection{calcul de beta}
    \paragraph{}
    ...\[ \mbox{Béta } \beta = \log( \frac{ \cosh( \frac{ \mbox{Ondulation} }{17,37} ) } { \sinh( \frac{ \mbox{Ondulation} }{17,37} ) } )\]
    \begin{lstlisting}
        /* beta */

        beta = Math.log( ( cosh( ondulation / 17.37 ) ) / ( sinh( ondulation / 17.37 ) ) );

    \end{lstlisting}

\subsection{calcul de gamma}
    \paragraph{}
    ...\[ \mbox{Gamma } \gamma = \sinh( \frac{ \beta }{ 2 * \mbox{Ordre} } ) \]
    \begin{lstlisting}
        /* gamma */

        gamma = sinh( beta / ( 2 * ordre ) );

    \end{lstlisting}

\subsection{calcul de R}
    \paragraph{}
    Si l'ordre est pair \[ \mbox{Résistance équivalente Rn} = ( \tanh{ \frac{ \beta }{ 4 } } ) ^2 * \mbox{Impédance} \]
    Si l'ordre est impair \[ \mbox{Résistance équivalente Rn} = \mbox{Impédance} \]
    \begin{lstlisting}
        /* calcul de R */

        if ( ( ordre % 2 ) != 0 )
        {
            R = 1;
        }
        else
        {
            R = tanh( beta / 4 ) * tanh( beta / 4 );
        }

        /* calcul de Rn */

        Rn = R * impedance;

    \end{lstlisting}

\subsection{calcul de Ak }
    \paragraph{}
    ... \[ \mbox{Ak} = ( \tanh{ \frac{ \beta }{ 4 } } ) ^2 * \mbox{Impédance} \]
    \begin{lstlisting}
        /* calcul des Ak */

        for( k = 1; k <= ordre; k++ )
        {
            Ak[k] = Math.sin( ( ( 2 * k-1 ) * Math.PI ) / ( 2 * ordre ) );
        }

    \end{lstlisting}

\subsection{calcul de Bk }
    \paragraph{}
    ... \[ \mbox{Bk} = 0 \]
    \begin{lstlisting}
        /* calcul des Bk */

        for( k = 1; k <= ordre; k++ )
        {
            Bk[k] = gamma * gamma + Math.sin( k * Math.PI / ordre ) * Math.sin( k * Math.PI / ordre );
        }

    \end{lstlisting}

\subsection{calcul de Gk }
    \paragraph{}
    ... \[ \mbox{Gk} = 0 \]
    \begin{lstlisting}
        /* calcul des Gk */

        Gk[1] = 2 * Ak[1] / gamma;

        for( k = 2; k <= ordre ; k++ )
        {
            Gk[k] = ( 4 * Ak[k-1] * Ak[k] ) / ( Bk[k-1] * Gk[k-1] );
        }

    \end{lstlisting}

\subsection{calcul de L }
    \paragraph{}
    ... \[ \mbox{L} = 0 \]
    \begin{lstlisting}
        /* calcul des L */

        for( k = 1; k <= ordre ; k++ )
        {
            l[k] = ( impedance * Gk[k] ) / Wc ;
        }

    \end{lstlisting}

\subsection{calcul de C }
    \paragraph{}
    ... \[ \mbox{C} = 0 \]
    \begin{lstlisting}
        /* calcul des C */

        for( k = 1; k <= ordre ; k++ )
        {
            c[k] = Gk[k] / ( ( impedance * Wc ) );
        }

    \end{lstlisting}



\end{document}
