\documentclass[a4paper,11pt]{report}
\usepackage[utf8]{inputenc}
\usepackage{color,amsmath,xcolor,listings,graphicx}
\usepackage[francais]{babel}

%% paramétrage pour les zones de 'code'
\lstset{
    language=Perl, commentstyle=\textit, frame=shadowbox,
    rulesepcolor=\color{gray}, basicstyle=\ttfamily\small, columns=flexible,
    tabsize=3, extendedchars=true, showspaces=false,
    showstringspaces=false, numbers=left, numberstyle=\tiny,
    breaklines=true, breakautoindent=true, captionpos=b, morecomment=[l]{//}
%language=Octave %-> choose the language of the code
%basicstyle=\footnotesize %-> the size of the fonts used for the code
%numbers=left %-> where to put the line-numbers
%numberstyle=\footnotesize %-> size of the fonts used for the line-numbers
%stepnumber=2 -> the step between two line-numbers.
%numbersep=5pt -> how far the line-numbers are from the code
%backgroundcolor=\color{white} -> sets background color (needs package)
%showspaces=false -> show spaces adding particular underscores
%showstringspaces=false -> underline spaces within strings
%showtabs=false -> show tabs within strings through particular underscores
%frame=single -> adds a frame around the code
%tabsize=2 -> sets default tab-size to 2 spaces
%captionpos=b -> sets the caption-position to bottom
%breaklines=true -> sets automatic line breaking
%breakatwhitespace=false -> automatic breaks happen at whitespace
%morecomment=[l]{//} -> displays comments in italics (language dependent)
}


%% infos du document
\title{Filtres}
\author{Loïc Barbaresco, Rémi Barbaste, Robin Degironde, Émeric Tosi}
\date{\today}


\begin{document}

%% Afficher la page de garde : Titre + Auteur(s) + Date de dernière compilation
    \maketitle{}

    \setcounter{tocdepth}{1} % définir la profondeur de l'Index
    \renewcommand{\contentsname}{Sommaire} % renommer l'Index en Sommaire
    \tableofcontents{} % afficher l'Index
    \clearpage


%% Différentes Parties / Chapitres / Autres fichiers à inclure :

\chapter{Introduction}
        \paragraph{}
lolilol !

\chapter{Théorie}
    \section{calcul de 1}
        \paragraph{}

    \section{calcul de 2}
        \paragraph{}

    \section{calcul de 3}
        \paragraph{}

    \section{calcul de 4}
        \paragraph{}

    \section{calcul de ... }
        \paragraph{}


\chapter{Code}
           \lstset{
                language=java, basicstyle=\ttfamily\small, columns=flexible,
                tabsize=2, extendedchars=true, showspaces=false,
                showstringspaces=false, numbers=left, numberstyle=\tiny,
                breaklines=true, breakautoindent=true, captionpos=b
            }

    \begin{lstlisting}
        /* pulsation */
        Wc = 2 * Math.PI * freqCoup;
    \end{lstlisting}


    \begin{lstlisting}
        /* beta */
        beta = Math.log( ( cosh( ondulation / 17.37 ) ) / ( sinh( ondulation / 17.37 ) ) );
    \end{lstlisting}


    \begin{lstlisting}
        /* gamma */
        gamma = sinh( beta / ( 2 * ordre ) );
    \end{lstlisting}


    \begin{lstlisting}
        /* calcul de R */
        if ( ( ordre % 2 ) != 0 )
        {
            R = 1;
        }
        else
        {
            R = tanh( beta / 4 ) * tanh( beta / 4 );
        }
    \end{lstlisting}


    \begin{lstlisting}
        /* calcul de Rn */
        Rn = R * impedance;
    \end{lstlisting}


    \begin{lstlisting}
        /* calcul des Ak */
        for( k = 1; k <= ordre; k++ )
        {
            Ak[k] = Math.sin( ( ( 2 * k-1 ) * Math.PI ) / ( 2 * ordre ) );
        }
    \end{lstlisting}


    \begin{lstlisting}
        /* calcul des Bk */
        for( k = 1; k <= ordre; k++ )
        {
            Bk[k] = gamma * gamma + Math.sin( k * Math.PI / ordre ) * Math.sin( k * Math.PI / ordre );
        }
    \end{lstlisting}


    \begin{lstlisting}
        /* calcul des Gk */
        Gk[1] = 2 * Ak[1] / gamma;

        for( k = 2; k <= ordre ; k++ )
        {
            Gk[k] = ( 4 * Ak[k-1] * Ak[k] ) / ( Bk[k-1] * Gk[k-1] );
        }
    \end{lstlisting}


    \begin{lstlisting}
        /* calcul des L */
        for( k = 1; k <= ordre ; k++ )
        {
            l[k] = ( impedance * Gk[k] ) / Wc ;
        }
    \end{lstlisting}


    \begin{lstlisting}
        /* calcul des C */
        for( k = 1; k <= ordre ; k++ )
        {
            c[k] = Gk[k] / ( ( impedance * Wc ) );
        }
    \end{lstlisting}



\end{document}
