\documentclass[a4paper,11pt]{article}
\usepackage[utf8]{inputenc}
\usepackage{color,amsmath,xcolor,listings,graphicx}
\usepackage[francais]{babel}

%% paramétrage pour les zones de 'code'
\lstdefinelanguage{javascript}{
  keywords={typeof, new, true, false, catch, function, return, null, catch, switch, var, if, for, in, while, do, else, case, break},
  keywordstyle=\color{blue}\bfseries,
  ndkeywords={class, export, boolean, throw, implements, import, this},
  ndkeywordstyle=\color{darkgray}\bfseries,
  identifierstyle=\color{black},
  sensitive=false,
  comment=[l]{//},
  morecomment=[s]{/*}{*/},
  commentstyle=\color{purple}\ttfamily,
  stringstyle=\color{red}\ttfamily,
  morestring=[b]',
  morestring=[b]"
}

\lstset{
    language=javascript, commentstyle=\textit, frame=shadowbox,
    rulesepcolor=\color{gray}, basicstyle=\ttfamily\small, columns=flexible,
    tabsize=4, extendedchars=true, showspaces=false,
    showstringspaces=false, numbers=left, numberstyle=\tiny,
    breaklines=true, breakautoindent=true, captionpos=b, morecomment=[l]{//}
%language=Octave %-> choose the language of the code
%basicstyle=\footnotesize %-> the size of the fonts used for the code
%numbers=left %-> where to put the line-numbers
%numberstyle=\footnotesize %-> size of the fonts used for the line-numbers
%stepnumber=2 -> the step between two line-numbers.
%numbersep=5pt -> how far the line-numbers are from the code
%backgroundcolor=\color{white} -> sets background color (needs package)
%showspaces=false -> show spaces adding particular underscores
%showstringspaces=false -> underline spaces within strings
%showtabs=false -> show tabs within strings through particular underscores
%frame=single -> adds a frame around the code
%tabsize=2 -> sets default tab-size to 2 spaces
%captionpos=b -> sets the caption-position to bottom
%breaklines=true -> sets automatic line breaking
%breakatwhitespace=false -> automatic breaks happen at whitespace
%morecomment=[l]{//} -> displays comments in italics (language dependent)
}


%% infos du document
\title{Filtres}
\author{Loïc Barbaresco, Rémi Barbaste, Robin Degironde, Émeric Tosi}
\date{\today}


\begin{document}

%% Afficher la page de garde : Titre + Auteur(s) + Date de dernière compilation
    \maketitle{}

    \setcounter{tocdepth}{1} % définir la profondeur de l'Index
    \renewcommand{\contentsname}{Sommaire} % renommer l'Index en Sommaire
    \tableofcontents{} % afficher l'Index
    \clearpage


%% Différentes Parties / Chapitres / Autres fichiers à inclure :

\section{Introduction}
        \paragraph{}
Le filtrage analogique peut être réalisé à partir d’éléments passifs (RLC) ou actifs (structures à AOP + RC).
Il est basé sur la détermination d’un cahier des charges et du calcul d’une fonction de transfert associée.
Pour les signaux numériques, le filtrage numérique (manipulation d’équation en z -1) existe mais ne sera pas abordé dans ce cours.
Le but du filtrage est d’isoler une information utile contenue dans une bande passante déterminée, par rapport aux bruits et aux autres informations existants hors de cette bande.
Il s’agit donc d’un filtrage fréquentiel.
La décomposition dans le domaine de Fourier du signal et le travail sur son spectre sont donc dans un premier temps indispensables.

\section{inconvenients}
    \paragraph{}
Temps de propagation de groupe non constant en bande passante (déphasage non linéaire)
    \paragraph{}
Très mauvaise linéarité de phase
    \paragraph{}
Ondulation dans la bande passante

\section{caracteristiques}
    \paragraph{}
Ordre plus petit pour une grande sélectivité
    \paragraph{}
Pente d’atténuation supérieure à 40dB/décade autour de la fréquence de coupure
    \paragraph{}
À ordre n, il présente n ondulation(s) dans la bande passante
    \paragraph{}
Filtres d’ordre impair : Impédances d’entrée et de sortie identiques

\section{calcul de la pulsation}
    \paragraph{}
    ...\[ \mbox{Pulsation Wc} = 2 * \pi * \mbox{ Fréquence de coupure}\]
    \begin{lstlisting}
        /* pulsation */

        Wc = 2 * Math.PI * freqCoup;

    \end{lstlisting}

\section{calcul de beta}
    \paragraph{}
    ...\[ \mbox{Béta } \beta = \log( \frac{ \cosh( \frac{ \mbox{Ondulation} }{17,37} ) } { \sinh( \frac{ \mbox{Ondulation} }{17,37} ) } )\]
    \begin{lstlisting}
        /* beta */

        beta = Math.log( ( cosh( ondulation / 17.37 ) ) / ( sinh( ondulation / 17.37 ) ) );

    \end{lstlisting}

\section{calcul de gamma}
    \paragraph{}
    ...\[ \mbox{Gamma } \gamma = \sinh( \frac{ \beta }{ 2 * \mbox{Ordre} } ) \]
    \begin{lstlisting}
        /* gamma */

        gamma = sinh( beta / ( 2 * ordre ) );

    \end{lstlisting}

\section{calcul de R}
    \paragraph{}
    Si l'ordre est pair \[ \mbox{Résistance équivalente Rn} = ( \tanh{ \frac{ \beta }{ 4 } } ) ^2 * \mbox{Impédance} \]
    Si l'ordre est impair \[ \mbox{Résistance équivalente Rn} = \mbox{Impédance} \]
    \begin{lstlisting}
        /* calcul de R */

        if ( ( ordre % 2 ) != 0 )
        {
            R = 1;
        }
        else
        {
            R = tanh( beta / 4 ) * tanh( beta / 4 );
        }

        /* calcul de Rn */

        Rn = R * impedance;

    \end{lstlisting}

\section{calcul de Ak }
    \paragraph{}
    ... \[ \mbox{Ak} = ( \tanh{ \frac{ \beta }{ 4 } } ) ^2 * \mbox{Impédance} \]
    \begin{lstlisting}
        /* calcul des Ak */

        for( k = 1; k <= ordre; k++ )
        {
            Ak[k] = Math.sin( ( ( 2 * k-1 ) * Math.PI ) / ( 2 * ordre ) );
        }

    \end{lstlisting}

\section{calcul de Bk }
    \paragraph{}
    ... \[ \mbox{Bk} = 0 \]
    \begin{lstlisting}
        /* calcul des Bk */

        for( k = 1; k <= ordre; k++ )
        {
            Bk[k] = gamma * gamma + Math.sin( k * Math.PI / ordre ) * Math.sin( k * Math.PI / ordre );
        }

    \end{lstlisting}

\section{calcul de Gk }
    \paragraph{}
    ... \[ \mbox{Gk} = 0 \]
    \begin{lstlisting}
        /* calcul des Gk */

        Gk[1] = 2 * Ak[1] / gamma;

        for( k = 2; k <= ordre ; k++ )
        {
            Gk[k] = ( 4 * Ak[k-1] * Ak[k] ) / ( Bk[k-1] * Gk[k-1] );
        }

    \end{lstlisting}

\section{calcul de L }
    \paragraph{}
    ... \[ \mbox{L} = 0 \]
    \begin{lstlisting}
        /* calcul des L */

        for( k = 1; k <= ordre ; k++ )
        {
            l[k] = ( impedance * Gk[k] ) / Wc ;
        }

    \end{lstlisting}

\section{calcul de C }
    \paragraph{}
    ... \[ \mbox{C} = 0 \]
    \begin{lstlisting}
        /* calcul des C */

        for( k = 1; k <= ordre ; k++ )
        {
            c[k] = Gk[k] / ( ( impedance * Wc ) );
        }

    \end{lstlisting}



\end{document}
